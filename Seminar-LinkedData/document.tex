\documentclass{acm_proc_article-sp}

\newdef{definition}{Definition}
\newdef{example}{Example}
\usepackage{color}
\usepackage{listings}
\def\mymathhyphen{{\hbox{-}}}

\definecolor{blue}{rgb}{0.25,0.35,1}
\definecolor{green}{rgb}{0.15,0.8,0.15}
\lstset{
language=sql,
frame=single,
xleftmargin=0.15cm,
morecomment=[l]{@},
commentstyle=\color{green},  
basicstyle=\ttfamily\scriptsize\singlespacing,
keywordstyle=\color{blue}\bfseries,
stepnumber=1,
numbersep=10pt,
tabsize=4,
showspaces=false,
showstringspaces=false}


\begin{document}

\title{Schema enrichment in knowledge bases}
%\subtitle{[Extended Abstract]

\numberofauthors{1} 
\author{
\alignauthor
Tim Schmiedl\\
       \affaddr{Department of Computer Science, University Freiburg}\\
       \affaddr{Sundgaualle 52}\\
       \affaddr{73110 Freiburg}\\
       \email{timschmiedl@neptun.uni-freiburg.de}
}

\date{\today}

\maketitle

\begin{abstract}

The Semantic Web is steadily growing and the availability of large knowledge
graphs increased over the past years. In spite of the growing number of
knowledge bases, there exist very few with a sophisticated schema. Often they only consist of a
collection of facts with no consistent structure. Other knowledge bases contain
only schema information without instances.
\\
But only the combination of both, sophisticated schema and
available instance data can enable powerful reasoning, easier consistency
checking and improved query abilities.
\\
This article shows two methods for the semantic enrichment of large OWL
knowledge bases. The first method focuses at finding and creating class expressions
in an automatic or semiautomatic approach, based on the existing data in the
graph.
Whereas the second method enriches knowledge bases with different types of OWL2
axioms.
\end{abstract}


%%%%%%%%%%%%%%  INTRODUCTION  %%%%%%%%%%%%%%
\section{Introduction}
% evtl: Semantic Web Stack

- semantic web: growing, bigger knowledge graphs

- Open data Initiative, Protoge ontologie etc
-> hard to maintain, debug / find error inconsitencies

- lack sophisticated schema (only schema no instances, only facts)

- combination good schema + instance data  --> powerful reasoning, consistency,
improved query

- Example: Person birthplace
  + Benefits
  + missing info
  + semi-automated





% \begin{figure}
% \centering
% \epsfig{file=img/example1.png, width = 8cm}
% \caption{A sample black and white graphic (.eps format).}
% \end{figure}

\section{Main A}
Lorem ipsum dolor sit amet, consectetur adipiscing elit.
Quisque feugiat velit ut eros tempus hendrerit.
Curabitur pretium placerat turpis eget sollicitudin.
Quisque lacinia commodo enim et condimentum.
Sed pulvinar, enim non porta lacinia, purus arcu placerat ante, eget scelerisque ipsum dolor vel ex.
Etiam interdum erat eget ultricies euismod.
Pellentesque ultrices lectus sapien, sit amet rutrum urna cursus quis.
Aenean interdum luctus nisl, id finibus urna.
Ut et lobortis enim.
Phasellus pharetra elit enim, nec scelerisque ipsum vestibulum at.
Morbi bibendum odio elementum, tristique orci eget, ultricies justo.
Morbi pretium consectetur quam, a malesuada nunc vehicula in.
Mauris a maximus lacus.


Curabitur iaculis arcu velit, at fermentum odio cursus id.
Mauris laoreet porta augue et tempus.
Vestibulum mollis tristique risus.
Mauris mollis, massa sed sagittis mollis, nibh eros rhoncus nulla, eu posuere metus dui vel risus.
Nunc elementum erat eget sapien condimentum, sed ultrices orci finibus.
Donec justo metus, mollis egestas iaculis a, luctus eget augue.
Suspendisse quis venenatis elit, vitae venenatis tortor.
Suspendisse quis enim lacus.
Curabitur a posuere ex, vitae mattis leo.
Integer id convallis lacus, sit amet tristique nisi.
Praesent malesuada vehicula pulvinar.
Donec ullamcorper convallis elit pharetra sodales.


Nunc condimentum lacus ipsum, et consequat enim porttitor eu.
In hac habitasse platea dictumst.
Pellentesque id fringilla sem.
Pellentesque egestas ligula eu mauris volutpat finibus eget et magna.
Proin fringilla dolor nec congue malesuada.
Suspendisse lacinia molestie nibh, non laoreet lectus hendrerit ac.
Aliquam vitae nunc ipsum.
Cras sollicitudin tempus risus, eu varius elit ultrices nec.
Etiam vel lorem aliquam, gravida magna pellentesque, mattis enim.
Class aptent taciti sociosqu ad litora torquent per conubia nostra, per inceptos himenaeos.
Donec malesuada, metus quis eleifend laoreet, nisi quam aliquam ante, at vestibulum neque diam quis eros.
Quisque vel leo sit amet neque auctor dignissim.
Vestibulum vel purus id lorem porttitor finibus.
Vestibulum ante ipsum primis in faucibus orci luctus et ultrices posuere cubilia Curae;



\section{Main B}
Lorem ipsum dolor sit amet, consectetur adipiscing elit.
Quisque feugiat velit ut eros tempus hendrerit.
Curabitur pretium placerat turpis eget sollicitudin.
Quisque lacinia commodo enim et condimentum.
Sed pulvinar, enim non porta lacinia, purus arcu placerat ante, eget scelerisque ipsum dolor vel ex.
Etiam interdum erat eget ultricies euismod.
Pellentesque ultrices lectus sapien, sit amet rutrum urna cursus quis.
Aenean interdum luctus nisl, id finibus urna.
Ut et lobortis enim.
Phasellus pharetra elit enim, nec scelerisque ipsum vestibulum at.
Morbi bibendum odio elementum, tristique orci eget, ultricies justo.
Morbi pretium consectetur quam, a malesuada nunc vehicula in.
Mauris a maximus lacus.


Curabitur iaculis arcu velit, at fermentum odio cursus id.
Mauris laoreet porta augue et tempus.
Vestibulum mollis tristique risus.
Mauris mollis, massa sed sagittis mollis, nibh eros rhoncus nulla, eu posuere metus dui vel risus.
Nunc elementum erat eget sapien condimentum, sed ultrices orci finibus.
Donec justo metus, mollis egestas iaculis a, luctus eget augue.
Suspendisse quis venenatis elit, vitae venenatis tortor.
Suspendisse quis enim lacus.
Curabitur a posuere ex, vitae mattis leo.
Integer id convallis lacus, sit amet tristique nisi.
Praesent malesuada vehicula pulvinar.
Donec ullamcorper convallis elit pharetra sodales.


Nunc condimentum lacus ipsum, et consequat enim porttitor eu.
In hac habitasse platea dictumst.
Pellentesque id fringilla sem.
Pellentesque egestas ligula eu mauris volutpat finibus eget et magna.
Proin fringilla dolor nec congue malesuada.
Suspendisse lacinia molestie nibh, non laoreet lectus hendrerit ac.
Aliquam vitae nunc ipsum.
Cras sollicitudin tempus risus, eu varius elit ultrices nec.
Etiam vel lorem aliquam, gravida magna pellentesque, mattis enim.
Class aptent taciti sociosqu ad litora torquent per conubia nostra, per inceptos himenaeos.
Donec malesuada, metus quis eleifend laoreet, nisi quam aliquam ante, at vestibulum neque diam quis eros.
Quisque vel leo sit amet neque auctor dignissim.
Vestibulum vel purus id lorem porttitor finibus.
Vestibulum ante ipsum primis in faucibus orci luctus et ultrices posuere cubilia Curae;



\section{Main C}
Lorem ipsum dolor sit amet, consectetur adipiscing elit.
Quisque feugiat velit ut eros tempus hendrerit.
Curabitur pretium placerat turpis eget sollicitudin.
Quisque lacinia commodo enim et condimentum.
Sed pulvinar, enim non porta lacinia, purus arcu placerat ante, eget scelerisque ipsum dolor vel ex.
Etiam interdum erat eget ultricies euismod.
Pellentesque ultrices lectus sapien, sit amet rutrum urna cursus quis.
Aenean interdum luctus nisl, id finibus urna.
Ut et lobortis enim.
Phasellus pharetra elit enim, nec scelerisque ipsum vestibulum at.
Morbi bibendum odio elementum, tristique orci eget, ultricies justo.
Morbi pretium consectetur quam, a malesuada nunc vehicula in.
Mauris a maximus lacus.


Curabitur iaculis arcu velit, at fermentum odio cursus id.
Mauris laoreet porta augue et tempus.
Vestibulum mollis tristique risus.
Mauris mollis, massa sed sagittis mollis, nibh eros rhoncus nulla, eu posuere metus dui vel risus.
Nunc elementum erat eget sapien condimentum, sed ultrices orci finibus.
Donec justo metus, mollis egestas iaculis a, luctus eget augue.
Suspendisse quis venenatis elit, vitae venenatis tortor.
Suspendisse quis enim lacus.
Curabitur a posuere ex, vitae mattis leo.
Integer id convallis lacus, sit amet tristique nisi.
Praesent malesuada vehicula pulvinar.
Donec ullamcorper convallis elit pharetra sodales.


Nunc condimentum lacus ipsum, et consequat enim porttitor eu.
In hac habitasse platea dictumst.
Pellentesque id fringilla sem.
Pellentesque egestas ligula eu mauris volutpat finibus eget et magna.
Proin fringilla dolor nec congue malesuada.
Suspendisse lacinia molestie nibh, non laoreet lectus hendrerit ac.
Aliquam vitae nunc ipsum.
Cras sollicitudin tempus risus, eu varius elit ultrices nec.
Etiam vel lorem aliquam, gravida magna pellentesque, mattis enim.
Class aptent taciti sociosqu ad litora torquent per conubia nostra, per inceptos himenaeos.
Donec malesuada, metus quis eleifend laoreet, nisi quam aliquam ante, at vestibulum neque diam quis eros.
Quisque vel leo sit amet neque auctor dignissim.
Vestibulum vel purus id lorem porttitor finibus.
Vestibulum ante ipsum primis in faucibus orci luctus et ultrices posuere cubilia Curae;
 

\section{Related Work}
Related work can be divided into two categories: the first part covers
supervised machine learning with OWL, the second part is focused on
(semi-)automated ontology engineering methods.
\\
Early work of supervised learning in description logic was published in
\cite{related_9,related_10}, which uses the so called \emph{least common
subsumer} to solve the learning problem. Later work invented the concept of the
refinement operator to solve the problem in a top-down
approach.\cite{related_7} The refinement operator was later adapted for
description logic \cite{refinement1,refinement2,refinement3} and is used as described in
CELOE.
\\
The starting point of (semi-)automatic ontology engineering was set by
\cite{related_31}, a formal concept analysis was descibed in \cite{related_5}.
Another interesting approach is presented in \cite{related_relexo} which
proposes to improve knowlege bases through relation exploration. It is
implemented in the RELExO framework.

\section{Conclusions}
- more

- more

- more


% The following two commands are all you need in the
% initial runs of your .tex file to
% produce the bibliography for the citations in your paper.
\bibliographystyle{abbrv}
\bibliography{sigproc}  % sigproc.bib is the name of the Bibliography in this case
% You must have a proper ".bib" file
%  and remember to run:
% latex bibtex latex latex
% to resolve all references
%
% ACM needs 'a single self-contained file'!

\balancecolumns
% That's all folks!
\end{document}
