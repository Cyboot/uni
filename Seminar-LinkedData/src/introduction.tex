\section{Introduction}
While the availability and size of knowledge bases has grown exponentially over
the last few years, there is often a significantly lack of sophisticated
schema.
Enrichment of schema information, based on already existing data, is the main
subject of this article.
To illustrate the benefits of a sophisticated schema together with reliable
instance data, consider the following example.

\begin{example}
Imagine a knowledge base containing famous persons like Barack Obama, Mahatma
Gandhi, Elvis Presley etc. and a property called \texttt{birthPlace}. An
algorithm may find out that every of the person instances contain this property
and therefore \texttt{birthPlace} can be seen as a functional property with the
domain \texttt{Person} and the range \texttt{Place}.\\
\vspace{-0.90 cm}
\begin{verbatim}
OBjectProperty: birthPlace
  Domain: Person
  Range: Place
  SubPropertyOf: hasBeenAt
\end{verbatim}
\end{example}
\vspace{-0.4 cm}

Adding such an axiom to the knowledge base can have several benefits:
(1) Axioms can serve as documentation for the right usage of the
schema, (2) adding additional schema can improve consistency and help to debug missing or incorrect
information, (3) additional (implicit) information can be inferred. In our
example you can observe that every person also \texttt{hasBeenAt} their
\texttt{birthPlace}.
Adding new axioms and finding/fixing missing information can be done in a
semi-automated approach. A knowledge engineer can decide if new axioms are
correct and improve the semantic schema. He than can add the axiom and adjust
potentially incorrect or missing data. \cite{paper2}

In summary, the combination of a solid schema with large instance data allows
powerful reasoning, improved query ability and simplified consistency checking. 