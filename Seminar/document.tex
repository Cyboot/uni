\documentclass{proseminar}

\begin{document}

\conferenceinfo{Albert-Ludwigs Universit\"at Freiburg\\Technische Fakult\"at, Institut f\"ur Informatik\\Lehrstuhl f\"ur Datenbanken \& Informationssysteme}{}

\title{Trust and Privacy in Social Media \\
\huge Content oriented trust}

\numberofauthors{1} 
\author{
Tim Schmiedl\\
\email{tim.schmiedl@neptun.uni-freiburg.de}
}

\maketitle

\section{Introduction}
Microblogging is a well-established paradigm for interaction in online social networks.
 real-time fashion
 mobile internet devices
 propagating news and information about developing events

\subsection*{Current Situation}
 Besides helping to communicate relevant events on a day-
to-day basis, microblogging can be particularly helpful during emergency and/or crisis
situations

provide
real-time information from the actual location where the crisis is unfolding. This
information often spreads faster and to a wider audience than what traditional news
media sources can achieve.

\subsection*{Problems}
variety of content; NEWS / Chat
 valuable to the user and its immediate circle offriends VS  valuable to a broader community
NO DISTINCTION between news / chat


\subsection*{Goals of the Papers}
In this work we focus mostly on the credibility of newsworthy information
is a correlation between how information propagates and the credibility 

aid them in the process of discovering reliable information.  achieved in an automatic way using features
extracted from information cascades

\section{Twitter Case Study}


\section{Retrieving and Labeling of Data}


\section{Ranking of Tweets}

\section{Prediction Model for Tweets}

\section{Evaluation}

\section{Related Work}

\section{Conclusion}



\bibliographystyle{abbrv}
\bibliography{bibliography}  

\balancecolumns

\end{document}
