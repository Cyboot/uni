\documentclass{article}
\usepackage{relsize}

\usepackage[utf8]{inputenc}
\usepackage{fancyhdr}
\usepackage{amsmath}
\usepackage{listings}
\usepackage{graphicx}
 
\pagestyle{fancy}
\fancyhf{}
\lhead{Exercise Sheet 5}
\rhead{Tim Schmiedl}
\rfoot{Page \thepage}

\lstset{ 
    language=SQL, % choose the language of the code
    basicstyle=\footnotesize,
    frame=single, 
    keywordstyle=\bfseries, % style for keywords
}

\begin{document}
\title{Data Analysis and Query Languages \\
 Exercise Sheet 5}
\date{\today}
\author{Tim Schmiedl} 
\maketitle

%%%%%%%%%%%%%%%%%%%%%%%%%%%%%%%%%%%%%%%%%%%%%%%%%%%%%%%%%%%%%
%%%%%%%%%%%%%%%%%%%%%%%%%%%%%%%%%%%%%%%%%%%%%%%%%%%%%%%%%%%%%
\section*{Exercise 1}

$(R\bowtie^{1,2,3'}_{3=1'})^*$:\\
\\
\begin{tabular}{|c|c|c|}
\hline
1 & 2 & 3'\\
\hline
St. Andrews & Bus Op 1 & London\\
\hline
St. Andrews & Bus Op 1 & Brussels\\
\hline
Edinburgh & Train Op 1 & Brussels\\
\hline
Train Op 1 & part\_of & NatExpress\\
\hline
\end{tabular}\\
\\
\\
$(\bowtie^{1',2,3}_{1=2'}R)^*$:\\
\\
\begin{tabular}{|c|c|c|}
\hline
1' & 2 & 3\\
\hline
St. Andrews & part\_of & NatExpress\\
\hline
Edinburgh & part\_of & EastCost\\
\hline
London & part\_of & Eurostar\\
\hline
\end{tabular}\\
\\
\\
Reach $_{\to}$ defined by Left Kleene Closure: $(\bowtie^{1',2,3}_{3'=1}R)^*$\\

\vspace{2cm}
%%%%%%%%%%%%%%%%%%%%%%%%%%%%%%%%%%%%%%%%%%%%%%%%%%%%%%%%%%%%%
%%%%%%%%%%%%%%%%%%%%%%%%%%%%%%%%%%%%%%%%%%%%%%%%%%%%%%%%%%%%%
\section*{Exercise 2}



\vspace{2cm}
%%%%%%%%%%%%%%%%%%%%%%%%%%%%%%%%%%%%%%%%%%%%%%%%%%%%%%%%%%%%%
%%%%%%%%%%%%%%%%%%%%%%%%%%%%%%%%%%%%%%%%%%%%%%%%%%%%%%%%%%%%%
\section*{Exercise 3}
\subsection*{a)}
For evaluation results see \textbf{I, II, III} in excercise \textbf{c)}.\\\\
Generally speaking the values for MAE and RMSE seems to improve for higher
\textit{MinSimilarity}. This seems to be logical as now only more similar
items/users are considered.\\
Also Pearson-correlation seems to work better than cosine similarity.

\subsection*{b)}
u: user with the rating to predict\\
i: neighbors of user u\\
maxN: maximum of neighbors to consider \\
\\
sim(u, i): similarity between u and i\\
avg(u): average rating of user u\\
rating(u): rating for user u\\
\\
$
rating(u) = avg(u) + \frac{\sum\limits_{i=0}^{maxN} (rating(i) - avg(i)) *
sim(u, i)} {\sum\limits_{i=0}^{maxN} |sim(u, i) | }
$

\subsection*{c)}
Changes see code.\\
\\
Generally the item based approach seems to work a little better, especially for
MAE and RMSE. Therefore see the following evaluation results.
\vspace{1cm}\\
\paragraph{I)}
Cosine=true, MinSim=0.8, MinOverlap=3, MinNeighbors=3, Neighbors=25\\
\begin{tabular}{| l | l | l | l |}
  \hline          
& UserBased & ItemBased  \\\hline
MAE & 0.801 & 0.792 \\
RMSE & 1.018 & 1.0 \\
Precision & 0.303 & 0.313 \\
Recall & 0.688 & 0.685 \\
  \hline  
\end{tabular}
\vspace{1cm}\\
\paragraph{II)}
Cosine=false, MinSim=0.2, MinOverlap=5, MinNeighbors=5, Neighbors=25\\
\begin{tabular}{| l | l | l | l |}
  \hline          
& UserBased & ItemBased  \\\hline
MAE & 0.795 & 0.754 \\
RMSE & 1.024 & 0.969 \\
Precision & 0.349 & 0.354 \\
Recall & 0.666 & 0.624 \\
  \hline  
\end{tabular}
\vspace{1cm}\\
\paragraph{III)}
Cosine=false, MinSim=0.9, MinOverlap=5, MinNeighbors=5, Neighbors=25\\
\begin{tabular}{| l | l | l | l |}
  \hline          
& UserBased & ItemBased  \\\hline
MAE & 0.177 & 0.71 \\
RMSE & 0.228 & 0.978 \\
Precision & 0.04 & 0.105 \\
Recall & 0.002 & 0.018 \\
  \hline  
\end{tabular}
\end{document}
